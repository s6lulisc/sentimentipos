\documentclass[11pt, a4paper, leqno]{article}
\usepackage{a4wide}
\usepackage[T1]{fontenc}
\usepackage[utf8]{inputenc}
\usepackage{float, afterpage, rotating, graphicx}
\usepackage{epstopdf}
\usepackage{longtable, booktabs, tabularx}
\usepackage{fancyvrb, moreverb, relsize}
\usepackage{eurosym, calc}
% \usepackage{chngcntr}
\usepackage{amsmath, amssymb, amsfonts, amsthm, bm}
\usepackage{caption}
\usepackage{mdwlist}
\usepackage{xfrac}
\usepackage{setspace}
\usepackage[dvipsnames]{xcolor}
\usepackage{subcaption}
\usepackage{minibox}
\usepackage{float}
% \usepackage{pdf14} % Enable for Manuscriptcentral -- can't handle pdf 1.5
% \usepackage{endfloat} % Enable to move tables / figures to the end. Useful for some of the
% submissions.

\usepackage[
    natbib=true,
    bibencoding=inputenc,
    bibstyle=authoryear-ibid,
    citestyle=authoryear-comp,
    maxcitenames=3,
    maxbibnames=10,
    useprefix=false,
    sortcites=true,
    backend=biber
]{biblatex}
\AtBeginDocument{\toggletrue{blx@useprefix}}
\AtBeginBibliography{\togglefalse{blx@useprefix}}
\setlength{\bibitemsep}{1.5ex}
\addbibresource{../../paper/refs.bib}

\usepackage[unicode=true]{hyperref}
\hypersetup{
    colorlinks=true,
    linkcolor=black,
    anchorcolor=black,
    citecolor=NavyBlue,
    filecolor=black,
    menucolor=black,
    runcolor=black,
    urlcolor=NavyBlue
}


\widowpenalty=10000
\clubpenalty=10000

\setlength{\parskip}{1ex}
\setlength{\parindent}{0ex}
\setstretch{1.5}


\begin{document}

\title{Sentiment Analysis Scores and IPO First Day Returns\thanks{Luke Liscio and Leonardo Rota Sperti, University of Bonn. Emails: \href{mailto:s6lulisc@uni-bonn.de}{\nolinkurl{s6lulisc [at] uni-bonn [dot] de}} \href{mailto:s6lerota@uni-bonn.de}{\nolinkurl{s6lerota [at] uni-bonn [dot] de}}}}

\author{Luke Liscio and Leonardo Rota Sperti}

\date{
    {\bf Preliminary -- please do not quote}
    \\[1ex]
    \today
}

\maketitle


\begin{abstract}
    The purpose of this repository is to conduct an empirical study using sentiment analysis.
The analysis is performed on US financial news articles related to companies that had Initial Public Offerings in the year 2018.
We investigate the relationship between the polarity score of chosen companies and their first day returns.
Polarity score measures the sentiment of a text by assigning positive or negative values to words and averaging those values to get an overall score.
In this analysis, first day returns is measured as the percentage change from opening price to closing price on the first day of trading.
This repo contains python scripts that handle the data management, analsis, and production of a paper with the findings from this analysis.
\end{abstract}

\clearpage


\section{Introduction} % (fold)
\label{sec:introduction}

The first data set for this project is taken from
\url{https://www.kaggle.com/datasets/jeet2016/us-financial-news-articles}.
The second data set is from
\url{https://www.iposcoop.com/scoop-track-record-from-2000-to-present}.
The first data set contains over 300,000 US financial news articles from 2018.
The second data set contains data on Initial Public Offerings in the United States from 2000 until 2020.
If you are using this template, please cite this item from the references:
\citet{GaudeckerEconProjectTemplates}.

\section{Figures and Tables}
\label{sec:Figures and Tables}

\begin{figure}[H]

    \centering
    \includegraphics[width=0.85\textwidth]{../python/figures/regression_plot}

    \caption{\emph{Python:} Plot of sentiment scores and first day returns.}
    \label{fig:python-predictions}

\end{figure}

\begin{table}[H]
    \input{../python/tables/summary_table.tex}
    \caption{\label{tab:python-summary}\emph{Python:} Estimation results of the
        linear regression.}
\end{table}




% section introduction (end)

% \clearpage

\setstretch{1}
\printbibliography
\setstretch{1.5}


% \appendix

% The chngctr package is needed for the following lines
% \counterwithin{table}{section}
% \counterwithin{figure}{section}

\end{document}
